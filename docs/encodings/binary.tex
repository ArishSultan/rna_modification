\section{Binary}
Binary encoding is a technique employed to transform DNA or RNA sequences into 
numerical representations that are amenable to computational and machine learning 
techniques. In this encoding scheme, each nucleotide (A, C, G, T or U) is mapped 
to a distinct binary vector, enabling computational models to process these biological 
sequences.

Each nucleotide is represented as a four-dimensional binary vector where exactly 
one position corresponds to 1 (indicating presence) and the other three positions 
correspond to 0 (indicating absence). Here is the typical representation:

\begin{lstlisting}[basicstyle=\ttfamily]
    - Adenine  (A) -> [1.0, 0.0, 0.0, 0.0]
    - Cytosine (C) -> [0.0, 1.0, 0.0, 0.0]
    - Guanine  (G) -> [0.0, 0.0, 1.0, 0.0]
    - Thymine  (T) -> [0.0, 0.0, 0.0, 1.0]
    - Uracil   (U) -> [0.0, 0.0, 0.0, 1.0]
\end{lstlisting}

\noindent
For a sequence of length $N$, represented as $S = \{s_1, s_2, \ldots, s_N\}$, the 
binary encoding process can be mathematically described as a function, 
$f: S \rightarrow E$, mapping the original sequence into a sequence of 
4-dimensional vectors, $E = \{e_1, e_2, \ldots, e_N\}$.\\

\noindent
\textbf{Advantages of Binary Encoding:}
\begin{itemize}
  \item It provides a clear, unambiguous representation of DNA or RNA sequences.
  \item It makes the sequence data suitable for various machine learning and data
  analysis methods.
  
  \item The process of encoding and decoding is straightforward and efficient.
\end{itemize}

\noindent
\textbf{Limitations of Binary Encoding:}
\begin{itemize}
  \item It doesn't capture the positional context or the correlation between 
  adjacent nucleotides in a sequence.
  \item The encoded representation may result in high-dimensional data, especially
   for long sequences.
  \item It provides a simple 1-to-1 mapping and does not capture the complexities
   of biological functions or mutations.
\end{itemize}

\noindent
Some additional details about Binary Encoding:

\begin{itemize}
  \item The choice of binary vector for each nucleotide is arbitrary, as long as
   each nucleotide has a unique representation.
  
  \item It can be combined with other methods (like position-specific encoding or
   k-mer counting) for a more comprehensive representation.
  
  \item For nucleotides that are not recognized, it is common to encode them as 
  a zero vector $[0.0, 0.0, 0.0, 0.0]$.
\end{itemize}

\noindent
In conclusion, Binary Encoding is a foundational technique in bioinformatics that 
translates the intricate language of DNA or RNA sequences into a binary format 
suitable for computational analysis. Although it may not capture the full complexity 
of biological sequences, it forms a basis upon which more complex encoding schemes 
can be built.