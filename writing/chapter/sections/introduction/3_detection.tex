\section{Pseudouridine Detection Techniques}\label{sec:pseudouridine-detection-techniques}
  \subsection*{Chemical Detection}

    Detecting pseudouridine (\pseudo) in RNA is challenging due to its subtle structural differences from uridine.
    Several traditional experimental techniques require specialized equipment and manual labor, making them resource-intensive.
    These methods include:

    \begin{itemize}
      \item \textbf{High-Performance Liquid Chromatography (HPLC)}: Used for separation and identification of nucleoside modifications, including pseudouridine, but it is labor-intensive and unsuitable for large-scale studies~\cite{gehrke_quantitative_1982}.
      \item \textbf{Mass Spectrometry (MS)}: Provides highly accurate detection but requires complex protocols and specialized equipment, limiting its scalability for high-throughput studies~\cite{de_hoffmann_mass_2011}.
      \item \textbf{Next-Generation Sequencing-Based Methods}: Techniques such as Pseudo-seq and $\Psi$-seq use chemical derivatization to map pseudouridine across the transcriptome, but these methods are also complex and resource-intensive~\cite{carlile_pseudo-seq_2015}.
    \end{itemize}

  \subsection*{Computational Prediction}
    Machine learning (ML) and deep learning (DL) have become integral tools in bioinformatics, offering scalable and cost-effective methods for analyzing complex biological data.
    These computational approaches can process large datasets to identify patterns and make predictions that are impractical with traditional techniques.

    Key advantages of ML-based methods in bioinformatics include:

    \begin{itemize}
      \item \textbf{High-Throughput Data Analysis}: ML algorithms can handle vast amounts of biological data, enabling the analysis of genomic, transcriptomic, proteomic, and metabolomic datasets.
      They can identify patterns and associations that are not readily apparent, facilitating discoveries in areas such as gene expression profiling and variant detection~\cite{libbrecht_machine_2015}.

      \item \textbf{Feature Extraction and Predictive Modeling}: ML models can learn complex relationships by extracting relevant features and building predictive models.
      In bioinformatics, this allows for accurate prediction of protein structures, gene functions, and disease associations~\cite{chicco_machine_2020}.

      \item \textbf{Integration with Experimental Data}: Computational predictions complement experimental approaches by providing insights that guide experimental design.
      For example, ML models can prioritize candidate genes or pathways for further investigation, saving resources and focusing efforts where validation is most needed~\cite{larranaga_machine_2006}.

      \item \textbf{Scalability and Flexibility}: ML-based approaches are highly scalable and adaptable to various biological problems and datasets.
      They can be applied across different organisms and conditions, which is essential for studying complex biological systems and diseases~\cite{min_deep_2016}.

      \item \textbf{Continuous Improvement}: ML models can be continually refined as new data become available, improving their predictive accuracy over time.
      This iterative learning process is crucial in bioinformatics, where data is constantly being generated~\cite{esteva_guide_2019}.
    \end{itemize}
