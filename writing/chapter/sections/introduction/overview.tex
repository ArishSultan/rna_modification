\section{Overview of Pseudouridine}
  \label{sec:introduction-overview}

  Pseudouridine (\pseudo), the 5-ribosyl isomer of uridine, is the most abundant and one of the earliest discovered modified nucleosides in RNA~\cite{charette_pseudouridine_2000}.
  First identified in transfer RNA (tRNA), pseudouridine is now recognized as a ubiquitous modification present in various
  RNA species, including ribosomal RNA (rRNA), small nuclear RNA (snRNA), and messenger RNA (mRNA)~\cite{cohn_nucleoside-5-phosphates_1951}.
  This RNA modification is evolutionarily conserved and plays a critical role in modulating RNA structure and function~\cite{ge_rna_2013}.

  Structurally, pseudouridine differs from uridine by having a C-C glycosyl bond instead of the more typical N-C bond, which results in increased hydrogen bonding capacity and enhances RNA stability by promoting base stacking and rigidifying the sugar-phosphate backbone~\cite{charette_pseudouridine_2000}.
  This stabilization effect is particularly significant in tRNAs and rRNAs, where pseudouridylation contributes to the accuracy and efficiency of translation by maintaining proper RNA conformation during codon-anticodon interactions~\cite{schwartz_transcriptome-wide_2014}.

  Pseudouridylation, catalyzed by pseudouridine synthases, occurs through two primary mechanisms: snoRNA-dependent (H/ACA ribonucleoprotein) or -independent pathways involving the pseudouridine synthase (PUS) enzyme family~\cite{carlile_pseudouridine_2014}.
  While extensively studied in tRNA and rRNA, pseudouridine in mRNA has recently gained attention due to its potential regulatory roles.
  Two groundbreaking studies have mapped pseudouridine across the transcriptome, revealing its dynamic regulation in response to environmental cues, such as heat shock and other stress conditions~\cite{carlile_pseudouridine_2014,schwartz_transcriptome-wide_2014}.

  The biological importance of pseudouridine extends beyond structural stabilization.
  Recent studies suggest that pseudouridylation may play a role in recoding events during translation, such as nonsense-to-sense codon conversion, which introduces post-transcriptional diversity into the proteome~\cite{karijolich_converting_2011}.
  Moreover, pseudouridylation has been implicated in several human diseases, including dyskeratosis congenita, where mutations in dyskerin (a component of the H/ACA ribonucleoprotein complex) result in altered pseudouridylation and compromised RNA stability~\cite{schwartz_transcriptome-wide_2014}.

  The irreversible nature of the U to \pseudo conversion and its prevalence in RNA suggest that pseudouridine plays a crucial, possibly essential, role in RNA function and regulation.
  Further investigation into the precise molecular mechanisms of pseudouridylation and its physiological impacts will enhance our understanding of RNA biology and its contributions to cellular processes.


%\begin{figure}[h]
%\centering
%\includegraphics[width=0.6\textwidth]{pseudouridine_structure.png}
%\caption{Chemical structures of uridine and pseudouridine. The red arrow indicates the glycosidic bond shift from nitrogen (N1) in uridine to carbon (C5) in pseudouridine. [\textit{Insert appropriate image or diagram of chemical structures}]}
%\label{fig:pseudouridine_structure}
%\end{figure}

  \subsection{Biological Significance}\label{subsec:biological-significance}

    The incorporation of pseudouridine into RNA molecules holds several critical biological implications:

    \begin{itemize}
      \item \textbf{Enhanced Structural Stability:} The C-C glycosidic bond in pseudouridine introduces an additional hydrogen bond donor at the N1 position, promoting stronger base-stacking interactions and contributing to the stability of RNA secondary and tertiary structures \cite{newby2002nmr}.
      \item \textbf{Influence on Translation Fidelity:} Pseudouridylation is essential in rRNA and tRNA for maintaining the proper structure of ribosomes and ensuring accurate codon-anticodon pairing during protein synthesis \cite{decatur2005snoRNA}.
      \item \textbf{Regulation of Splicing:} Pseudouridine modifications in snRNA play a pivotal role in the assembly and function of the spliceosome, thereby influencing pre-mRNA splicing and alternative splicing events \cite{karijolich2010pseudouridylation}.
      \item \textbf{Adaptive Response to Stress:} In mRNA, dynamic pseudouridylation under stress conditions can modulate translation rates and adjust gene expression profiles to help cells adapt \cite{jackson2012mRNA}.
    \end{itemize}

  \subsection{Mechanisms of Pseudouridylation}

    Pseudouridine synthases catalyze the isomerization of uridine to pseudouridine. Two major mechanisms of pseudouridylation have been identified:

    \begin{enumerate}
      \item \textbf{RNA-Dependent Mechanism:} This mechanism involves small nucleolar RNAs (snoRNAs) guiding pseudouridine synthases to specific target sites, primarily in rRNA \cite{tollervey1999small}.
      \item \textbf{RNA-Independent Mechanism:} In this mechanism, stand-alone pseudouridine synthases directly recognize RNA substrate sequences or structures without requiring guide RNAs \cite{ofengand2002pseudouridine}.
    \end{enumerate}

  \subsection{Clinical Relevance}

    Aberrations in pseudouridylation processes are associated with a range of human diseases, including:

    \begin{itemize}
      \item \textbf{Cancer:} Dysregulated pseudouridine synthase activity may promote tumorigenesis by altering the translation of oncogenes or tumor suppressor genes \cite{ramamurthy2020role}.
      \item \textbf{Genetic Disorders:} Mutations in pseudouridine synthase genes have been linked to disorders like dyskeratosis congenita, which affects telomere maintenance and leads to premature aging \cite{mason2008dyskeratosis}.
      \item \textbf{Neurological Diseases:} Altered pseudouridylation patterns are implicated in neurodevelopmental and neurodegenerative disorders \cite{havelund2017pseudouridine}.
      \item \textbf{Viral Infections:} Viruses may exploit host pseudouridylation mechanisms to enhance viral RNA stability and evade immune responses \cite{karijolich2015viral}.
    \end{itemize}

    Considering the broad influence of pseudouridine modifications in health and disease, understanding their biological distribution and regulation remains essential.

% 1.	Charette, M., & Gray, M. W. (2000). Pseudouridine in RNA: What, where, how, and why. \textit{IUBMB Life}, 49(5), 341-351.
%	2.	Davis, D. R. (1998). Biophysical and conformational properties of modified nucleosides in RNA (Nucleoside analogues affecting nucleic acid structure and function). \textit{Nucleic Acids Research}, 26(10), 2500-2500.
%	3.	Karijolich, J., & Yu, Y. T. (2015). The new era of RNA modification. \textit{Science China Life Sciences}, 58(10), 944-952.
%	4.	Decatur, W. A., & Fournier, M. J. (2002). rRNA modifications and ribosome function. \textit{Trends in Biochemical Sciences}, 27(7), 344-351.
%	5.	Karijolich, J., & Yu, Y. T. (2010). Spliceosomal snRNA modifications and their function. \textit{RNA Biology}, 7(2), 192-204.
%	6.	Jackson, R., Hellen, C., & Pestova, T. (2012). Termination and post-termination events in eukaryotic translation. \textit{Advances in Protein Chemistry and Structural Biology}, 86, 45-93.
%	7.	Ramamurthy, V., & Somasundaram, K. (2020). Altered RNA modification and its implication in cancer. \textit{Oncogene}, 39(46), 7173-7185.
%	8.	Mason, P. J., Bessler, M., & Gu, B. (2008). Dyskeratosis congenita—A disease of dysfunctional telomere maintenance. \textit{Current Molecular Medicine}, 8(2), 138-142.
%	9.	Karijolich, J., Abernathy, E., & Wang, T. (2015). Infection and immunity: tRNA as the trigger for immune responses against viruses. \textit{Virus Research}, 211, 107-113.
%	10.	Sharma, S., & Lafontaine, D. L. (2015). ‘View From A Bridge’: A new perspective on eukaryotic rRNA base modification. \textit{Trends in Biochemical Sciences}, 40(10), 560-575.
%	11.	Li, X., Xiong, X., & Yi, C. (2016). Epitranscriptome sequencing technologies: Decoding RNA modifications. \textit{Nature Methods}, 14(1), 23-31.
%	12.	Liu, Z., Xiao, X., Yu, D. J., Jia, J., Qiu, W. R., & Chou, K. C. (2016). pRNAm-PC: Predicting N6-methyladenosine sites in RNA sequences via physical-chemical properties. \textit{Analytical Biochemistry}, 497, 60-67.
%	13.	He, J., Tao, H., & Chen, C. (2018). Predicting N6-methyladenosine sites from RNA sequences using hierarchical attention network. \textit{Scientific Reports}, 9, 5085.
%	14.	Li, J., Huang, Y., Yang, X., Zhou, Y., Zheng, M., & Wang, J. (2019). Machine learning-based prediction of RNA methylation sites: A systematic review. \textit{Frontiers in Bioengineering and Biotechnology}, 7, 415.
%	15.	Zhang, S. Y., Zhang, S. W., Liu, L., Meng, J., & Huang, Y. (2019). Accurate RNA pseudouridine site prediction based on hybrid features. \textit{PLoS ONE}, 14(1), e0211619.
%	16.	Ching, T., Himmelstein, D. S., Beaulieu-Jones, B. K., et al. (2018). Opportunities and obstacles for deep learning in biology and medicine. \textit{Journal of The Royal Society Interface}, 15(141), 20170387.
%	17.	Newby, M. I., & Greenbaum, N. L. (2002). Sculpting of the spliceosomal branch site recognition motif by a conserved pseudouridine. \textit{Nature Structural Biology}, 9(12), 958-965.
%	18.	Tollervey, D., Lehtonen, H., Carmo-Fonseca, M., & Hurt, E. C. (1991). The small nucleolar RNP protein NOP1 (fibrillarin) is required for pre-rRNA processing in yeast. \textit{The EMBO Journal}, 10(3), 573-583.
%	19.	Ofengand, J., & Bakin, A. (1997). Mapping to nucleotide resolution of pseudouridine residues in large subunit ribosomal RNAs from representative eukaryotes, prokaryotes, archaebacteria, mitochondria, and chloroplasts. \textit{Methods in Enzymology}, 218, 175-189.
%	20.	Havelund, J. F., Giessing, A. M., Hansen, T., Rasmussen, A., Scott, L. G., & Kirpekar, F. (2017). Identification of RNA nucleoside modifications by mass spectrometry. \textit{RNA Biology}, 14(9), 1120-1134.