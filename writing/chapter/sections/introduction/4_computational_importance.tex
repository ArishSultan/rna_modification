\section{Significance of Computational Techniques}\label{sec:significance-of-computational-techniques}
  Although clinical and experimental approaches, such as High-Performance Liquid Chromatography (HPLC), Mass Spectrometry (MS), and Next-Generation Sequencing (NGS)-based methods, are considered the gold standard for detecting pseudouridine, these techniques are often labor-intensive, costly, and time-consuming.
  Each of these methods requires specialized equipment, expert personnel, and careful handling of RNA samples, making them impractical for high-throughput or large-scale studies.

  \begin{itemize}
    \item \textbf{HPLC}: requires precise calibration and laborious sample preparation, making it both time- and labor-intensive, especially when dealing with multiple samples.
    \item \textbf{Disease Mechanisms}: provides highly accurate results, but the equipment is expensive, and the sample processing steps are complicated.
    This limits its accessibility for routine use or large-scale studies.
    \item \textbf{Novel Therapeutic Targets}: such as Pseudo-seq and $\Psi$-seq offer a transcriptome-wide mapping of pseudouridine but require chemical derivatization and expensive reagents, adding to the cost and time demands of these methods.
  \end{itemize}

  While these clinical approaches offer high accuracy, they are not scalable for genome-wide studies or large datasets due to their resource-intensive nature.

  \subsection*{Advantages of Computational Techniques}
    In contrast, computational techniques-particularly those leveraging machine learning (ML)-provide a highly scalable and cost-effective solution for pseudouridine site prediction.
    Despite their computational limitations in precision compared to clinical methods, ML-based approaches offer several advantages:

    \begin{itemize}
      \item \textbf{Efficiency and Scalability}: Computational models can analyze large RNA sequence datasets simultaneously.
      For instance, a single ML model trained on known pseudouridine sites can predict potential sites across thousands of RNA sequences in a fraction of the time it would take to manually process these samples in a lab.
      This scalability is crucial for genome-wide studies, where clinical methods would be impractical due to time and cost constraints.
      \item \textbf{Cost-Effectiveness}: While clinical methods require expensive reagents, specialized instruments, and expert operators, computational methods require only the initial setup of data and algorithms, making them far more cost-effective for large-scale studies.
      Once a computational model is built, it can be applied to any number of RNA sequences without significant additional cost.
      \item \textbf{Rapid Hypothesis Generation}: ML models allow researchers to generate predictions quickly, providing insights that can guide experimental validation.
      Instead of laboriously testing every RNA sequence experimentally, researchers can use ML predictions to prioritize sequences for clinical validation, significantly reducing the number of experiments needed.
      This synergy between computational and clinical approaches enhances research efficiency.
      \item \textbf{Simultaneous Analysis of Multiple Conditions}: ML models can be trained to detect patterns across diverse biological conditions (e.g., different cell types, stress conditions, or disease states).
      In contrast, traditional clinical methods would require running individual experiments for each condition, making it impractical to study pseudouridylation under multiple scenarios at once.
    \end{itemize}

    While clinical techniques are undeniably more accurate in detecting pseudouridine, computational methods offer a complementary approach that accelerates the discovery process, enabling rapid analysis of large datasets.
    By integrating computational predictions with experimental validation, researchers can efficiently explore pseudouridylation across different organisms, tissues, and disease contexts.
    In the next section we will discuss some of the available models that try to solve this problem to some extent in their own novel ways.
