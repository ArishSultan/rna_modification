\section{Importance of Pseudouridine}\label{sec:importance-of-pseudouridine}
  \subsection*{Biological Significance}
    Pseudouridine is formed through the isomerization of uridine residues in RNA, a reaction catalyzed by pseudouridine synthases \cite{carlile_pseudouridine_2014}.
    This modification plays several important roles:

    \begin{itemize}
      \item \textbf{Structural Stability}: The conversion of uridine to pseudouridine introduces a C–C glycosidic bond, enhancing base stacking and stabilizing RNA secondary and tertiary structures~\cite{ge_rna_2013}.
      \item \textbf{RNA Function}: Pseudouridine is widely found in transfer RNA (tRNA), ribosomal RNA (rRNA), small nuclear RNA (snRNA), and messenger RNA (mRNA). In tRNA and rRNA, pseudouridylation maintains proper folding and function, crucial for accurate protein synthesis and ribosome assembly~\cite{schwartz_transcriptome-wide_2014}.
      \item \textbf{Pre-mRNA Splicing}: In snRNA, pseudouridylation is essential for the proper functioning of the spliceosome, influencing pre-mRNA splicing and gene expression regulation \cite{karijolich_converting_2011}.
      \item \textbf{Regulation of mRNA Translation}: Recent studies suggest pseudouridine in mRNA plays a role in regulating translation and cellular stress responses~\cite{carlile_pseudouridine_2014}.
    \end{itemize}

  \subsection*{Clinical Impacts}
    Aberrant pseudouridylation has been implicated in various human diseases. Key clinical impacts include:

    \begin{itemize}
      \item \textbf{Cancer:} Dysregulated pseudouridine synthase activity may promote tumorigenesis by altering the translation of oncogenes or tumor suppressor genes \cite{ramamurthy2020role}.
      \item \textbf{Genetic Disorders:} Mutations in pseudouridine synthase genes have been linked to disorders like dyskeratosis congenita, which affects telomere maintenance and leads to premature aging \cite{mason2008dyskeratosis}.
      \item \textbf{Neurological Diseases:} Altered pseudouridylation patterns are implicated in neurodevelopmental and neurodegenerative disorders \cite{havelund2017pseudouridine}.
      \item \textbf{Viral Infections:} Viruses may exploit host pseudouridylation mechanisms to enhance viral RNA stability and evade immune responses \cite{karijolich2015viral}.
    \end{itemize}
