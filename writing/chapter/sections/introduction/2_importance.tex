\section{Importance of Pseudouridine}\label{sec:importance-of-pseudouridine}
  \subsection*{Biological Significance}
    Pseudouridine is formed through the isomerization of uridine residues in RNA, a reaction catalyzed by pseudouridine synthases~\cite{carlile_pseudouridine_2014}.
    This modification plays several important roles:

    \begin{itemize}
      \item \textbf{Structural Stability}: The conversion of uridine to pseudouridine introduces a C\textminus{C} glycosidic bond, enhancing base stacking and stabilizing RNA secondary and tertiary structures~\cite{ge_rna_2013}.
      \item \textbf{RNA Function}: Pseudouridine is widely found in transfer RNA (tRNA), ribosomal RNA (rRNA), small nuclear RNA (snRNA), and messenger RNA (mRNA). In tRNA and rRNA, pseudouridylation maintains proper folding and function, crucial for accurate protein synthesis and ribosome assembly~\cite{schwartz_transcriptome-wide_2014}.
      \item \textbf{Pre-mRNA Splicing}: In snRNA, pseudouridylation is essential for the proper functioning of the spliceosome, influencing pre-mRNA splicing and gene expression regulation~\cite{karijolich_converting_2011}.
      \item \textbf{Regulation of mRNA Translation}: Recent studies suggest pseudouridine in mRNA plays a role in regulating translation and cellular stress responses~\cite{carlile_pseudouridine_2014}.
    \end{itemize}

  \subsection*{Clinical Impacts}
    Aberrant pseudouridylation has been implicated in various human diseases.
    Key clinical impacts include:
    \begin{itemize}
      \item \textbf{Cancer:} Elevated pseudouridine levels have been associated with cancer development and progression, particularly in human prostate cancer.
      \item Pseudouridine's role in stabilizing RNA structure, refining tRNA activity, enhancing translation fidelity, and regulating mRNA coding contributes to its potential as a novel biomarker in prostate cancer progression to advanced disease ~\cite{stockert_pseudouridine_2021}.
      \item \textbf{Genetic Disorders:} Mutations in dyskerin, a key component in pseudouridylation, are linked to genetic disorders like dyskeratosis congenita, which affects telomere maintenance and leads to premature aging. \cite{garus_dyskerin_2021}.
      \item \textbf{Viral Infections:} SARS-CoV-2 exploits host pseudouridine synthases, such as PUS7 (Pseudouridine synthase 7), to bind and modify its RNA, which may enhance viral RNA stability and support replication \cite{giambruno_unveiling_2023}.
    \end{itemize}
