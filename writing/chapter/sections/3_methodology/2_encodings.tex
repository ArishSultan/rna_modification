\section{RNA Encodings}\label{sec:encodings}
  These are essential for translating RNA sequences into formats that are more suitable for computational analysis in machine learning models.
  Since RNA consists of sequences of nucleotides—adenine (A), cytosine (C), guanine (G), and uracil (U)—encoding these sequences into numerical representations allows for their effective use in various predictive models.
  Different encoding schemes capture different properties of RNA sequences, such as nucleotide composition, structural patterns, or biochemical characteristics, helping researchers to model complex biological processes more accurately.
  Currently, there are numerous encoding methods available, with over 20 popular encodings widely used in RNA-related research.
  These include basic one-hot encoding, k-mer encoding, and more advanced structural and biochemical encodings.
  By selecting the appropriate encoding scheme, researchers can enhance the performance of machine learning models for tasks such as RNA modification prediction or secondary structure analysis.
  Tools like iLearn~\cite{chen_ilearn_2020} and iLearnPlus~\cite{chen_ilearnplus_2021}, widely adopted in the literature, have been instrumental in generating diverse RNA encoding schemes and are used by almost every researcher in the field.
  Related studies, such as those by~\cite{chen_comprehensive_2020} and~\cite{wang_brief_2022}, have demonstrated the impact of various RNA encoding strategies on prediction accuracy and biological insights.

  \subsection{One-Hot Encoding}\label{subsec:binary}
    One-hot encoding is a straightforward method that assigns each nucleotide a unique binary vector of length four, representing the four possible nucleotides.
    The mapping is defined as:

    \[
      A \rightarrow [1, 0, 0, 0], \quad
      C \rightarrow [0, 1, 0, 0], \quad
      G \rightarrow [0, 0, 1, 0], \quad
      U \rightarrow [0, 0, 0, 1]
    \]
    Given a sequence $S = \{s_1, s_2, \dots, s_n\}$, where each $s_i$ is a nucleotide, the one-hot encoded matrix is represented as:
    \[
      O(S) = \begin{bmatrix}
               1 & 0 & 0 & 0 & \dots \\
               0 & 1 & 0 & 0 & \dots \\
               0 & 0 & 1 & 0 & \dots \\
               0 & 0 & 0 & 1 & \dots \\
      \end{bmatrix}_{n \times 4}
    \]
    This matrix has $n$ rows, corresponding to the number of nucleotides in the sequence, and 4 columns, representing the four possible nucleotides.

    In many machine learning applications, the 2D matrix is flattened into a 1D vector of length $4n$:
    \[
      O_{flat}(S) = [1, 0, 0, 0, 0, 1, 0, 0, \dots]
    \]
    This flattened vector is particularly useful in traditional machine learning algorithms that expect fixed-length input vectors.
    In contrast, deep learning models such as convolutional neural networks (CNNs) often use the 2D matrix format to capture spatial dependencies within the sequence.

  \subsection{Nucleotide Density (ND)}\label{subsec:nd}
    Nucleotide Density (ND) encoding is a numerical representation that calculates the relative frequency of each nucleotide within a fixed-size window along the RNA sequence.
    This method captures local nucleotide composition and is particularly useful for understanding the nucleotide distribution in specific regions of RNA sequences.

    Given an RNA sequence $S = \{s_1, s_2, \dots, s_n\}$, where $s_i \in \{A, C, G, U\}$, the nucleotide density at position $i$ for nucleotide $X \in \{A, C, G, U\}$ is defined as the proportion of $X$ in the subsequence $S[1:i]$ (i.e., from the first nucleotide up to $s_i$). This can be expressed mathematically as:
    \[
      ND_X(i) = \frac{\sum_{j=1}^{i} \mathbb{I}(s_j = X)}{i}
    \]
    where $\mathbb{I}(s_j = X)$ is an indicator function that equals 1 if $s_j = X$ and 0 otherwise.
    The function calculates the cumulative frequency of nucleotide $X$ up to position $i$, divided by the index $i$.

    The final nucleotide density encoding for the entire sequence $S$ is a vector of length $n$:
    \[
      ND(S) = [ND_{s_1}(1), ND_{s_2}(2), \dots, ND_{s_n}(n)]
    \]
    where $ND_{s_i}(i)$ is the density of nucleotide $s_i$ at position $i$.

  \subsection{Nucleotide chemical property (NCP)}\label{subsec:ncp}
    Nucleotide Chemical Property (NCP) encoding represents RNA sequences by assigning a vector of three predefined chemical properties to each nucleotide (A, C, G, U). The chemical properties are simplified into binary values representing attributes such as molecular properties. This encoding allows machine learning models to utilize a simplified chemical feature representation of each nucleotide.

    Given an RNA sequence $S = \{s_1, s_2, \dots, s_n\}$, where $s_i \in \{A, C, G, U\}$, each nucleotide is represented by a three-dimensional binary vector encoding specific properties. For a nucleotide $s_i$, the encoding is determined as:
    \[
      NCP(s_i) = \begin{cases}
      [1.0, 1.0, 1.0]
                   , & \text{if } s_i = A \\
                   [0.0, 1.0, 0.0], & \text{if } s_i = C \\
                   [1.0, 0.0, 0.0], & \text{if } s_i = G \\
                   [0.0, 0.0, 1.0], & \text{if } s_i = U
      \end{cases}
    \]

    The final NCP encoding for the sequence $S$ is the concatenation of the vectors for each nucleotide, resulting in a vector of length $3n$:
    \[
      NCP(S) = [NCP(s_1), NCP(s_2), \dots, NCP(s_n)]
    \]

  \subsection{Nucleotide Composition Encoding}\label{subsec:nucleotide-composition-encoding}
    Nucleotide composition encodings are used to capture the frequency of occurrence of nucleotides within RNA sequences.
    These encodings can be applied to individual sequences, providing a representation of the nucleotide content within each sequence, or they can be applied to a list of sequences to give an aggregated view.
    This technique helps to preserve the compositional information of RNA sequences, which can then be used for various downstream tasks, such as machine learning models in RNA modification detection.

    \subsubsection{Nucleotide Composition (NC)}\label{subsubsec:nc}
      Nucleotide Composition (NC) encoding calculates the overall frequency of each nucleotide (A, C, G, U) within a single RNA sequence.
      This encoding provides a holistic view of the nucleotide makeup by counting the occurrences of each nucleotide and dividing by the total length of the sequence.

      Given an RNA sequence $S = \{s_1, s_2, \dots, s_n\}$, where each $s_i \in \{A, C, G, U\}$, the nucleotide composition for each nucleotide $X \in \{A, C, G, U\}$ is defined as:
      \[
        NC_X = \frac{\sum_{i=1}^{n} \mathbb{I}(s_i = X)}{n}
      \]
      where $\mathbb{I}(s_i = X)$ is an indicator function that equals 1 if $s_i = X$ and 0 otherwise.
      This formula counts the occurrences of nucleotide $X$ in the sequence and divides it by the total number of nucleotides $n$.

      The final NC encoding is a 4-dimensional vector:
      \[
        NC(S) = [NC_A, NC_C, NC_G, NC_U]
      \]

    \subsubsection{Dinucleotide Composition (DNC)}\label{subsubsec:dnc}
      Dinucleotide Composition (DNC) encoding captures the frequency of consecutive pairs of nucleotides in the sequence.
      Each pair of nucleotides from $\{A, C, G, U\}$ is counted, giving a total of 16 possible dinucleotide combinations (AA, AC, AG, AU \dots, UU).

      Given an RNA sequence $S = \{s_1, s_2, \dots, s_n\}$, the dinucleotide composition for each dinucleotide pair $XY$ is calculated as:
      \[
        DNC_{XY} = \frac{\sum_{i=1}^{n-1} \mathbb{I}(s_i = X \text{ and } s_{i+1} = Y)}{n-1}
      \]
      where $X, Y \in \{A, C, G, U\}$.
      The final DNC encoding is a 16-dimensional vector representing the frequency of each dinucleotide pair.

    \subsubsection{Tri-nucleotide Composition (TNC)}\label{subsubsec:tnc}
      Tri-nucleotide Composition (TNC) encoding captures the frequency of consecutive triplets of nucleotides in the sequence.
      With four possible nucleotides (A, C, G, U), there are 64 possible combinations of triplets (AAA, AAC, AAG \dots, UUU).

      For an RNA sequence $S = \{s_1, s_2, \dots, s_n\}$, the trinucleotide composition for each triplet $XYZ$ is defined as:
      \[
        TNC_{XYZ} = \frac{\sum_{i=1}^{n-2} \mathbb{I}(s_i = X \text{ and } s_{i+1} = Y \text{ and } s_{i+2} = Z)}{n-2}
      \]
      where $X, Y, Z \in \{A, C, G, U\}$.
      The final TNC encoding is a 64-dimensional vector representing the frequency of each trinucleotide triplet.

    \subsubsection{Pseudo-nucleotide Composition (PseKNC)}\label{subsubsec:pseknc}
      Pseudo-nucleotide Composition (PseKNC) encoding incorporates not only the frequency of nucleotides or nucleotide pairs but also the sequence order information.
      It introduces a set of correlation factors, which capture the relationships between distant nucleotides along the sequence, adding a layer of sequence-order information to the standard nucleotide composition methods.

      For an RNA sequence $S = \{s_1, s_2, \dots, s_n\}$, the PseKNC encoding is calculated by first computing a set of correlation factors $\theta_l$ for $l$-th order correlations, where $l$ refers to the distance between nucleotides being compared. The final PseKNC encoding is a combination of the original nucleotide composition along with these correlation factors:
      \[
        PseKNC(S) = [NC_A, NC_C, NC_G, NC_U, \theta_1, \theta_2, \dots, \theta_L]
      \]
      where $L$ represents the highest correlation order considered.
      The exact calculation of $\theta_l$ may depend on the specific method used but generally involves computing the correlation between nucleotides $l$ positions apart in the sequence.

  \subsection{Position-Specific Nucleotide Propensity}\label{subsec:position-specific-nucleotide-propensity}
    Position-specific propensity encodings capture how the frequency of nucleotides, dinucleotides, or trinucleotides varies at each position of the RNA sequence between two predefined groups (e.g., positive and negative samples). These methods provide insights into position-specific variations in nucleotide composition between groups. Below are the mathematical representations for nucleotide, dinucleotide, and trinucleotide propensities.

    \subsubsection{Position-Specific Nucleotide Propensity (PSNP)}\label{subsubsec:PSNP}
      Position-Specific Nucleotide Propensity (PSNP) encodes the difference in the occurrence of individual nucleotides (A, C, G, U) at each position between two groups of RNA sequences.
      This is useful for determining how each nucleotide behaves at specific positions across different groups.

      Given an RNA sequence $S = \{s_1, s_2, \dots, s_n\}$, let $P_j$ represent the frequency matrix for nucleotides in the positive group and $N_j$ for the negative group.
      For each position $j$, the nucleotide propensity for nucleotide $s_j$ is calculated as:
      \[
        PSNP_j(s_j) = \frac{P_j(s_j)}{|P|} - \frac{N_j(s_j)}{|N|}
      \]
      where $P_j(s_j)$ and $N_j(s_j)$ are the counts of nucleotide $s_j$ at position $j$ in the positive and negative groups, respectively, $|P|$ and $|N|$ are the total counts of nucleotides at position $j$ in each group.

      The final encoding for the sequence $S$ is a vector of length $n$, representing the propensity values for nucleotides across the sequence:
      \[
        PSNP(S) = [PSNP_1(s_1), PSNP_2(s_2), \dots, PSNP_n(s_n)]
      \]

    \subsubsection{Position-Specific Dinucleotide Propensity (PSDP)}\label{subsubsec:PSDP}
      Position-Specific Dinucleotide Propensity (PSDP) encodes the difference in dinucleotide (2-mers) frequencies between two groups of sequences at each position.
      This method captures the dinucleotide-level variations between positive and negative datasets, providing a richer level of detail than PSNP.

      Given an RNA sequence $S = \{s_1, s_2, \dots, s_n\}$, for each position $j$, define the dinucleotide $D_j = s_j s_{j+1}$.
      Let $P_j$ and $N_j$ be the position-specific dinucleotide frequency matrices for the positive and negative groups, respectively.
      The propensity for dinucleotide $D_j$ at position $j$ is defined as:
      \[
        PSDP_j(D_j) = \frac{P_j(D_j)}{|P|} - \frac{N_j(D_j)}{|N|}
      \]
      where $P_j(D_j)$ and $N_j(D_j)$ are the counts of dinucleotide $D_j$ at position $j$, $|P|$ and $|N|$ are the total counts of dinucleotides at that position.

      The final encoding for sequence $S$ is a vector of length $n-1$, representing the propensity values for dinucleotides across the sequence:
      \[
        PSDP(S) = [PSDP_1(D_1), PSDP_2(D_2), \dots, PSDP_{n-1}(D_{n-1})]
      \]

    \subsubsection{Position-Specific Trinucleotide Propensity (PSTNPss)}\label{subsubsec:pstnpss}
      Position-Specific Trinucleotide Propensity (PSTNPss) calculates the difference in trinucleotide frequencies between two predefined groups (e.g., positive and negative samples).
      This encoding captures the behavior of trinucleotide sequences at specific positions in RNA sequences by comparing their occurrence in positive and negative datasets.

      Given an RNA sequence $S = \{s_1, s_2, \dots, s_n\}$, the sequence is divided into overlapping trinucleotides (3-mers) such that for each position $j$ in the sequence (excluding the last two positions), the trinucleotide $T_j$ is defined as:
      \[
        T_j = s_j s_{j+1} s_{j+2}
      \]

      Let $P_j$ be the position-specific trinucleotide frequency matrix for the positive group, and $N_j$ for the negative group.
      For each trinucleotide $T_j$ at position $j$, the propensity encoding is computed as:
      \[
        PSTNPss_j(T_j) = \frac{P_j(T_j)}{|P|} - \frac{N_j(T_j)}{|N|}
      \]
      where $|P|$ and $|N|$ represent the total count of all trinucleotides at position $j$ in the positive and negative groups, respectively.

      The final encoding for sequence $S$ is a vector of length $n-2$, representing the propensity values for trinucleotides across the sequence:
      \[
        PSTNPss(S) = [PSTNPss_1(T_1), PSTNPss_2(T_2), \dots, PSTNPss_{n-2}(T_{n-2})]
      \]
