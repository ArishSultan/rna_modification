\chapter{Related Work}\label{ch:related-work}
  In this section, we summarize the methodologies and findings of various studies related to the prediction of pseudouridine sites in RNA sequences.
  Each study is described in terms of its encoding schemes, feature selection methods, machine learning algorithms, and the accuracy or performance metrics achieved.
  Finally, we present a comparative analysis of the studies in a summary table.


  \section{Past Studies}\label{sec:past-studies}

    \subsection*{iRNA-PseU \cite{chen_irna-pseu_nodate}}\label{subsec:iRNA-PseU}
      This study employed a feature encoding scheme based on nucleotide chemical properties and nucleotide frequency.
      The chemical properties (e.g., purine/pyrimidine structure, amino/keto groups, and hydrogen bond strength) were encoded alongside nucleotide occurrence frequency.
      This aimed to capture the biological relevance of nucleotide sequences for pseudouridine site prediction.

      \begin{itemize}
        \item \textbf{Feature Encoding}: NCP~\ref{enc:NCP}, PseKNC~\ref{subsubsec:pseknc}.
        \item \textbf{Model}: SVM~\ref{model:SVM} with RBF kernel.
      \end{itemize}

    \subsection*{PseUI \cite{he_pseui_2018}}\label{subsec:PseUI}
      This study combined various feature encoding schemes, including nucleotide composition (NC), dinucleotide composition (DC), pseudo dinucleotide composition (pseDNC), position-specific nucleotide propensity (PSNP), and position-specific dinucleotide propensity (PSDP). It employed sequential forward feature selection (SFS) to create a compact, discriminative feature set for pseudouridine site prediction.

      \begin{itemize}
        \item \textbf{Feature Encoding}: NC~\ref{subsubsec:nc}, DNC~\ref{subsubsec:dnc}, PseDNC~\ref{enc:PseDNC}, PSNP~\ref{enc:PSNP}, PSDP~\ref{enc:PSDP}.
        \item \textbf{Feature Selection}: SFS~\ref{fs:SFS}, evaluated with MCC~\ref{subsec:mcc}.
        \item \textbf{Model}: SVM~\ref{model:SVM}, jackknife cross-validation.
      \end{itemize}

    \subsection*{XG-PseU \cite{liu_xg-pseu_2020}}\label{subsec:XG-PseU}
      This study used multiple feature encoding schemes, such as nucleotide composition (NC), dinucleotide composition (DNC), trinucleotide composition (TNC), nucleotide chemical properties (NCP), nucleotide density (ND), and one-hot encoding.
      A two-step feature selection (forward selection and increment feature selection) was employed for optimal performance in pseudouridine site prediction.

      \begin{itemize}
        \item \textbf{Feature Encoding}: NC~\ref{subsubsec:nc}, DNC~\ref{subsubsec:dnc}, TNC~\ref{subsubsec:tnc}, NCP~\ref{enc:NCP}, ND~\ref{subsec:nd}, one-hot~\ref{subsec:binary}.
        \item \textbf{Feature Selection}: Forward feature selection~\ref{fs:FFS} and IFS~\ref{fs:IFS}.
        \item \textbf{Model}: XGBoost~\ref{model:Xgboost}, 10-fold cross-validation.
      \end{itemize}

    \subsection*{iPseU-NCP \cite{nguyen-vo_ipseu-ncp_2019}}\label{subsec:ipseu_ncp}
      This study utilized Random Forest (RF) with nucleotide chemical property (NCP) encoding to represent the structural aspects of RNA sequences.
      Feature importance within the Random Forest model was used for feature selection.

      \begin{itemize}
        \item \textbf{Feature Encoding}: NCP~\ref{enc:NCP}, PseKNC~\ref{subsubsec:pseknc}, CKSNAP~\ref{subsubsec:pseknc}.
        \item \textbf{Feature Selection}: Feature importance ranking in RF
        \item \textbf{Model}: RF~\ref{model:RF}, grid search with 5-fold cross-validation.
      \end{itemize}

    \subsection*{PseU-FKeERF \cite{chen_fuzzy_2024}}\label{subsec:PseU-FKeERF}
      This study utilized fuzzy kernel evidence Random Forest (FKeERF) with several RNA sequence encoding schemes to identify pseudouridine sites.
      Fuzzy logic was used to expand the feature space, and feature selection was performed through fuzzy mean clustering and Gaussian fuzzy membership.

      \begin{itemize}
        \item \textbf{Feature Encoding}: Binary~\ref{subsec:binary}, PSTNPss~\ref{subsec:pstnpss}, NCP~\ref{enc:NCP}, PseKNC~\ref{subsubsec:pseknc}.
        \item \textbf{Feature Selection}: Fuzzy mean clustering and Gaussian fuzzy membership.
        \item \textbf{Model}: Fuzzy C-mean~\ref{model:fuzzy-c}, Evidential Random Forest~\ref{model:ERF}, cross-validation and independent tests.
      \end{itemize}


  \section{Comparison of Related Work}\label{sec:comparison-of-related-work}
    The table below presents a comparison of the encoding schemes, feature selection methods, machine learning algorithms, and performance metrics for the studies reviewed in this chapter.

    \begin{table}[h!]
      \centering
      \begin{tabular}{lcccc}
        \toprule
        \textbf{Study}                           & \textbf{Acc} (\%) & \textbf{MCC}    & \textbf{Sp} (\%) & \textbf{Sn} (\%) \\
        \midrule
        iRNA-PseU\cite{chen_irna-pseu_nodate}    & 60.40             & 0.21            & 59.80            & 61.01            \\
        PseUI\cite{he_pseui_2018}                & 64.24             & 0.2849          & 63.64            & 64.85            \\
        iPseU-CNN\cite{tahir_ipseu-cnn_nodate}   & 66.68             & 0.34            & 68.78            & 65.00            \\
        XG-PseU\cite{liu_xg-pseu_2020}           & 66.05             & 0.32            & 68.65            & 63.45            \\
        iPseU-NCP\cite{nguyen-vo_ipseu-ncp_2019} & 62.92             & 0.24            & 65.05            & 58.79            \\
        FKeERF\cite{chen_fuzzy_2024}             & 78.08 (91.99)     & 0.5644 (0.8504) & 78.66 (94.97)    & 77.62 (90.11)    \\
        \bottomrule
      \end{tabular}
      \caption{Comparison of all Past studies on Homo sapiens dataset}
      \label{tab:human_comp_table}
    \end{table}

    \begin{table}[h!]
      \centering
      \begin{tabular}{lcccc}
        \toprule
        \textbf{Study}                           & \textbf{Acc} (\%) & \textbf{MCC} & \textbf{Sp} (\%) & \textbf{Sn} (\%) \\
        \midrule
        iRNA-PseU\cite{chen_irna-pseu_nodate}    & 69.07             & 0.38         & 64.83            & 73.31            \\
        PseUI\cite{he_pseui_2018}                & 70.44             & 0.4103       & 66.31            & 74.58            \\
        iPseU-CNN\cite{tahir_ipseu-cnn_nodate}   & 71.81             & 0.44         & 69.11            & 74.79            \\
        XG-PseU\cite{liu_xg-pseu_2020}           & 71.10             & 0.43         & 76.30            & 65.92            \\
        iPseU-NCP\cite{nguyen-vo_ipseu-ncp_2019} & 71.82             & 0.44         & 76.27            & 67.37            \\
        FKeERF\cite{chen_fuzzy_2024}             & 77.87             & 0.5571       & 77.82            & 77.85            \\
        \bottomrule
      \end{tabular}
      \caption{Comparison of all Past studies on Mus musculus dataset}
      \label{tab:mouse_comp_table}
    \end{table}

    \begin{table}[h!]
      \centering
      \begin{tabular}{lcccc}
        \toprule
        \textbf{Study}                           & \textbf{Acc} (\%) & \textbf{MCC}    & \textbf{Sp} (\%) & \textbf{Sn} (\%) \\
        \midrule
        iRNA-PseU\cite{chen_irna-pseu_nodate}    & 64.49             & 0.29            & 64.33            & 64.65            \\
        PseUI\cite{he_pseui_2018}                & 65.92             & 0.3185          & 66.88            & 64.97            \\
        iPseU-CNN\cite{tahir_ipseu-cnn_nodate}   & 68.15             & 0.37            & 70.45            & 66.36            \\
        XG-PseU\cite{liu_xg-pseu_2020}           & 73.42             & 0.47            & 69.48            & 77.35            \\
        iPseU-NCP\cite{nguyen-vo_ipseu-ncp_2019} & 69.59             & 0.40            & 62.10            & 77.07            \\
        FKeERF\cite{chen_fuzzy_2024}             & 87.73 (94.5)      & 0.7589 (0.8898) & 89.98 (96.08)    & 86.08 (93.29)    \\
        \bottomrule
      \end{tabular}
      \caption{Comparison of all Past studies on Saccharomyces cerevisiae dataset}
      \label{tab:yeast_comp_table}
    \end{table}
