\chapter{Related Work}\label{ch:related-work}
  In this section, we summarize the methodologies and findings of various studies related to the prediction of pseudouridine sites in RNA sequences.
  Each study is described in terms of its encoding schemes, feature selection methods, machine learning algorithms, and the accuracy or performance metrics achieved.
  Finally, we present a comparative analysis of the studies in a summary table.


  \section{}\label{sec:}

    \subsection{PPUS\cite{li_ppus_2015}}\label{subsec:ppus}
      \subsubsection*{Approach}
        This study employed [describe encoding schemes], where [brief explanation of the encoding method].
      \subsubsection*{Feature Selection}
        The feature selection method used was [method name], which [brief explanation].
      \subsubsection*{Machine Learning Algorithm}
        The machine learning model implemented was [ML algorithm], which [brief explanation of how the algorithm was applied].


  \section{Comparison of Related Work}\label{sec:comparison-of-related-work}
    The table below presents a comparison of the encoding schemes, feature selection methods, machine learning algorithms, and performance metrics for the studies reviewed in this chapter.

    \begin{table}[h!]
      \centering
      \begin{tabular}{lcccc}
        \toprule
        \textbf{Study} & \textbf{Acc} (\%) & \textbf{MCC} & \textbf{Sp} (\%) & \textbf{Sn} (\%) \\
        \midrule
        iRNA-PseU      & 60.40             & 0.21         & 59.80            & 61.01            \\
        \bottomrule
      \end{tabular}
      \caption{Comparison of all Past studies on Homo sapiens dataset}
      \label{tab:human_comp_table}
    \end{table}

    \begin{table}[h!]
      \centering
      \begin{tabular}{lcccc}
        \toprule
        \textbf{Study} & \textbf{Acc} (\%) & \textbf{MCC} & \textbf{Sp} (\%) & \textbf{Sn} (\%) \\
        \midrule
        iRNA-PseU      & 69.07             & 0.38         & 64.83            & 73.31            \\
        \bottomrule
      \end{tabular}
      \caption{Comparison of all Past studies on Mus musculus dataset}
      \label{tab:mouse_comp_table}
    \end{table}

    \begin{table}[h!]
      \centering
      \begin{tabular}{lcccc}
        \toprule
        \textbf{Study} & \textbf{Acc} (\%) & \textbf{MCC} & \textbf{Sp} (\%) & \textbf{Sn} (\%) \\
        \midrule
        iRNA-PseU      & 64.49             & 0.29         & 64.33            & 64.65            \\
        \bottomrule
      \end{tabular}
      \caption{Comparison of all Past studies on Saccharomyces cerevisiae dataset}
      \label{tab:yeast_comp_table}
    \end{table}


  \section{Conclusion}
    In conclusion, this chapter provided a detailed review of the related studies and their methodologies for predicting pseudouridine sites in RNA sequences. The comparison table highlights the similarities and differences between the approaches, offering insights into the performance of various techniques in this field.